% v2-acmsmall-sample.tex, dated March 6 2012
% This is a sample file for ACM small trim journals
%
% Compilation using 'acmsmall.cls' - version 1.3 (March 2012), Aptara Inc.
% (c) 2010 Association for Computing Machinery (ACM)
%
% Questions/Suggestions/Feedback should be addressed to => "acmtexsupport@aptaracorp.com".
% Users can also go through the FAQs available on the journal's submission webpage.
%
% Steps to compile: latex, bibtex, latex latex
%
% For tracking purposes => this is v1.3 - March 2012

\documentclass[prodmode,acmtecs]{acmsmall} % Aptara syntax

\usepackage{graphicx}
\usepackage{caption}
\usepackage{subfigure}
\usepackage{setspace}
\usepackage{tabulary}
\usepackage{fancybox}

% font in listings
\usepackage{courier}

\makeatletter
\newenvironment{CenteredBox}{% 
\begin{Sbox}}{% Save the content in a box
\end{Sbox}\centerline{\parbox{\wd\@Sbox}{\TheSbox}}}% And output it centered
\makeatother

\let\proof\relax
\let\endproof\relax
\usepackage{amsthm}
\newcommand{\specialcell}[2][c]{%
  \begin{tabular}[#1]{@{}c@{}}#2\end{tabular}}

\usepackage{lineno}
\usepackage{xfrac}

% Theorems, definitions, lemmas, etc
\newtheorem{Def}{Definition}
\newtheorem{Prop}{Proposition}

\usepackage{alltt}
\renewcommand{\ttdefault}{txtt}

\usepackage{listings}
\lstset{language=C, breaklines=true, mathescape}

\usepackage[cmex10]{amsmath}
\usepackage{url}



% Metadata Information
%\acmVolume{9}
%\acmNumber{4}
%\acmArticle{39}
%\acmYear{2010}
%\acmMonth{3}

% Document starts
\begin{document}

% Page heads
\markboth{F. Luporini et al.}{Optimal Finite Element Integration}

% Title portion
\title{Optimal Finite Element Integration}
\author{Fabio Luporini
\affil{Imperial College London}
David A. Ham
\affil{Imperial College London}
Paul H.J. Kelly
\affil{Imperial College London}}

\begin{abstract}
...
\end{abstract}

\category{G.1.8}{Numerical Analysis}{Partial Differential Equations -
  Finite element methods}

\category{G.4}{Mathematical Software}{Parallel and vector implementations}

\terms{Design, Performance}

\keywords{Finite element integration, local assembly, compilers, optimizations, SIMD vectorization}

\acmformat{Fabio Luporini, David A. Ham, and Paul   H. J. Kelly, 2015. Optimal Finite Element Integration.}

% At a minimum you need to supply the author names, year and a title.
% IMPORTANT: Full first names whenever they are known, surname last,
% followed by a period.  In the case of two authors, 'and' is placed
% between them.  In the case of three or more authors, the serial
% comma is used, that is, all author names except the last one but
% including the penultimate author's name are followed by a comma, and
% then 'and' is placed before the final author's name.  If only first
% and middle initials are known, then each initial is followed by a
% period and they are separated by a space.  The remaining information
% (journal title, volume, article number, date, etc.) is
% 'auto-generated'.

\begin{bottomstuff}

This research is partly funded by the MAPDES project, by the
Department of Computing at Imperial College London, by EPSRC through
grants EP/I00677X/1, EP/I006761/1, and EP/L000407/1, by NERC grants
NE/K008951/1 and NE/K006789/1, by the U.S.  National Science
Foundation through grants 0811457 and 0926687, by the U.S. Army
through contract W911NF-10-1-000, and by a HiPEAC collaboration
grant. The authors would like to thank Dr. Carlo Bertolli,
Dr. Lawrence Mitchell, and Dr. Francis Russell for their invaluable
suggestions and their contribution to the Firedrake project.

Author's addresses: Fabio Luporini $\&$ Paul H. J. Kelly, Department of Computing,
Imperial College London; David A. Ham, Department of Computing and
Department of Mathematics, Imperial College London; 
\end{bottomstuff}

\maketitle


\section{Introduction, Motivations, Related Work}

The need for rapidly implementing high performance, robust, and portable finite element methods has led to approaches based on automated code generation. This has been proved successful in the context of the FEniCS~\cite{Fenics} and Firedrake~\cite{firedrake-code} projects, which have become increasingly popular over the last years. In these frameworks, the weak variational form of a problem is expressed at high-level by means of a domain-specific language. The mathematical specification is manipulated and then passed to a form compiler, which generates a representation of local assembly operations. These operations numerically evaluate problem-specific integrals in order to compute so called local matrices and vectors, which represent the contributions from each element in the discretized domain to the equation solution. Local assembly code must be high performance: as the complexity of a variational form increases, in terms of number of derivatives, pre-multiplying functions, and polynomial order of the chosen function spaces, the resulting assembly kernels become more and more computationally expensive, covering a significant fraction of the overall computation run-time. 

Producing high performance implementations is, however, non-trivial. The complexity of mathematical expressions involved in the numerical integration, which varies from problem to problem, and the small size of the loop nest in which such integral is computed obstruct the optimization process. Traditional vendor compilers, such as \emph{GNU's} and \emph{Intel's}, fail at exploiting the structure inherent assembly expressions. Polyhedral-model-based source-to-source compilers, for instance~\cite{PLUTO}, mainly apply aggressive loop optimizations, such as tiling, but these are not particularly helpful in our context. This lack of suitable optimizing tools has led to the development of a number of higher-level approaches to maximize the performance of local assembly kernels.  In~\cite{quadrature1}, it is shown how automated code generation can be leveraged to introduce domain-specific optimizations, which a user cannot be expected to write ``by hand''. \cite{Kirby-FEM-opt} and~\cite{Francis} have studied, instead, different optimization techniques based on a mathematical reformulation of finite element integration. In~\cite{Luporini}, we have made one step forward by showing that different forms, on different platforms, require distinct sets of transformations if close-to-peak performance must be reached, and that low-level, domain-aware code transformations are essential to maximize instruction-level parallelism and register locality. The problem of optimizing local assembly routines has been tackled recently also for GPU architectures, for instance in~\cite{petsc-integration-gpu},~\cite{Klockner}, and~\cite{Bana}.

Our research has resulted in ....
... we build on our previous work~\cite{Luporini} ... and present a ... We argue that for complex, realistic forms, peak performance can be achieved only by  ... 
... also low-level optimisation ...
...also for the first time we provide a formal explanation of the math in terms of compiler theory

This is all implemented in COFFEE, which in turn is integrated with the Firedrake framework. We provide an extensive and unprecedented performance evaluation across a number of forms of increasing complexity, including some based on complex (hyperelasticity) models. We characterize our problems by varying polynomial order of the employed function spaces and number of pre-multiplying functions. To clearly distinguish the improvement achieved by this work, we will compare, for each test case, X sets of code variants: 1) unoptimized code, i.e. a local assembly routine as returned from the form compiler; 2) code optimized by FEniCS, i.e. the work in~\cite{quadrature1}; 3) code optimized as described in~\cite{Luporini}; ....



\section{Preliminaries}
\label{sec:background}

%is the computation of contributions of a specific cell in the discretized domain to the linear system which yields the PDE solution. The process consists of numerically evaluating problem-specific integrals to produce a matrix and a vector [Olgaard and Wells 2010; AMCG 2010], whose sizes depend on the order of the method. This operation is applied to all cells in the discretized domain. In this work we focus on local matrices, or “element matrices”, which are more costly to compute than element vectors.

We review finite element integration and possible implementations using notation and examples adopted in~\cite{quadrature1} and~\cite{Francis}. 

We consider the weak formulation of a linear variational problem
\begin{equation}
\begin{split}
Find\ u\ \in U\ such\ that \\
a(u, v) = L(v), \forall v \in V
\end{split}
\end{equation}
where $a$ and $L$ are, respectively, a bilinear and a linear form. The set of \textit{trial} functions $U$ and the set of \textit{test} functions $V$ are discrete function spaces. For simplicity, we assume $U = V$ and $\lbrace \phi_i \rbrace$ be the set of basis functions spanning $U$. The unknown solution $u$ can be approximated as a linear combination of the basis functions $\lbrace \phi_i \rbrace$. From the solution of the following linear system it is possible to determine a set of coefficients to express $u$
\begin{equation}
A\textbf{u} = b
\end{equation}
in which $A$ and $b$ discretize $a$ and $L$ respectively:
\begin{equation}
\centering
\begin{split}
A_{ij} = a(\phi_i(x), \phi_j(x)) \\
b_i = L(\phi_i(x))
\end{split}
\end{equation}
The matrix $A$ and the vector $b$ are computed in the so called assembly phase. Then, in a subsequent phase, the linear system is solved, usually by means of an iterative method, and $\textbf{u}$ is eventually evaluated. 

We focus on the assembly phase, which is often characterized as a two-step procedure: \textit{local} and \textit{global} assembly. Local assembly is the subject of the paper: this is about computing the contributions that an element in the discretized domain provide to the approximated solution of the equation. Global assembly, on the other hand, is the process of suitably ``inserting'' such contributions in $A$ and $b$. 

Without loss of generality, we illustrate local assembly in a concrete example; that is, the evaluation of the local element matrix for a Laplacian operator. Consider the weighted Laplace equation
\begin{equation}
- \nabla \cdot (w \nabla u) = 0
\end{equation}
in which $u$ is unknown, while $w$ is prescribed. The bilinear form associated with the weak variational form of the equation is:
\begin{equation}
a(v, u) = \int_\Omega w \nabla v \cdot \nabla u\ \mathrm{d}x
\end{equation}
The domain $\Omega$ of the equation is partitioned into a set of cells (elements) $T$ such that $\bigcup T = \Omega$ and $\bigcap T = \emptyset$. By defining $\lbrace \phi_i^K \rbrace$ as the set of local basis functions spanning $U$ on the element $K$, we can express the local element matrix as
\begin{equation}
\label{stiffness}
A_{ij}^K = \int_K w \nabla \phi_i^K \cdot \nabla \phi_j^K\ \mathrm{d}x
\end{equation}
The local element vector $L$ can be determined in an analogous way. 
%From the computational perspective, its evaluation is however less expensive than that of $A$.

\subsection{Quadrature Mode}
Quadrature schemes are conveniently used to numerically evaluate $A_{ij}^K$. For convenience, a reference element $K_0$ and an affine mapping $F_K : K_0 \rightarrow K$ to any element $K \in T$ are introduced. This implies a change of variables from reference coordinates $X_0$ to real coordinates $x = F_K (X_0)$ is necessary any time a new element is evaluated. The numerical integration routine based on quadrature representation over an element $K$ can be expressed as follows
\begin{equation}
\label{eq:quadrature}
\scriptsize
A_{ij}^K = \sum_{q=1}^N \sum_{\alpha_3=1}^n \phi_{\alpha_3}(X^q)w_{\alpha_3} \sum_{\alpha_1=1}^d \sum_{\alpha_2=1}^d \sum_{\beta=1}^d \frac{\partial X_{\alpha_1}}{\partial x_{\beta}} \frac{\partial \phi_i^K(X^q)}{\partial X_{\alpha_1}} \frac{\partial X_{\alpha_2}}{\partial x_{\beta}} \frac{\partial \phi_j^K(X^q)}{\partial X_{\alpha_2}} det F_K' W^q
\end{equation}
where $N$ is the number of integration points, $W^q$ the quadrature weight at the integration point $X^q$, $d$ is the dimension of $\Omega$, $n$ the number of degrees of freedom associated to the local basis functions, and $det$ the determinant of the Jacobian matrix used for the aforementioned change of coordinates.  

% Magari la summation sui coefficients la sputo dentro? che dici?

\subsection{Tensor Contraction Mode}
Starting from Equation~\ref{eq:quadrature}, exploiting basic mathematical properties we can rewrite the expression as
\begin{equation}
\label{eq:tensor}
\scriptsize
A_{ij}^K = \sum_{\alpha_1=1}^d \sum_{\alpha_2=1}^d \sum_{\alpha_3=1}^n det F_K' w_{\alpha_3} \sum_{\beta=1}^d \frac{X_{\alpha_1}}{\partial x_{\beta}} \frac{\partial X_{\alpha_2}}{\partial x_{\beta}} \int_{K_0} \phi_{\alpha_3} \frac{\partial \phi_{i_1}}{\partial X_{\alpha_1}} \frac{\partial \phi_{i_2}}{\partial X_{\alpha_2}} dX.
\end{equation}
A generalization of this transformation has been proposed in~\cite{Kirby:TC}. By only involving reference element terms, the integral in the equation can be pre-evaluated and stored in a temporary. The evaluation of the local tensor can then be abstracted as
\begin{equation}
A_{ij}^K = \sum_{\alpha} A_{i_1 i_2 \alpha}^0 G_{K}^\alpha
\end{equation}
in which the pre-evaluated "reference tensor" $A_{i_1 i_2 \alpha}$ and the cell-dependent "geometry tensor" $G_{K}^\alpha$ are exposed. 

\subsection{Qualitative Comparison}
Depending on the form being considered, the relative performance of the two modes, in terms of number of operations executed, can vary even quite dramatically. The presence of derivatives or coefficient functions in the input form tends to increase the size of the geometry tensor, making the traditional quadrature mode increasingly more indicate for "complex" forms. On the other hand, speed ups from adopting tensor mode can be significant in a wide class of forms in which the geometry tensor remains "sufficiently small". 

These two modes have been implemented in the Fenics Form Compiler. In this compiler, a simple heuristic is used to choose the most suitable mode for a given form. It consists of analysing each monomial in the form, counting the number of derivatives and coefficient functions, and checking if this number is greater than a constant found empirically~\cite{Fenics}. We will later comment on the efficacy of this approach (Section~\ref{sec:optimal-syntesis}. For the moment, we just recall that one of the goals of this research is to produce an intelligent system that is capable of selecting the optimal mode at the monomials level -- no heuristics, no "global" choice of the mode for all monomials in the form -- without affecting the cost of code generation.

\section{Optimality of Loop Nests}
\label{sec:optimal-impl}

In this section, we characterize our definition of optimality and we describe the assumptions under which it holds. 

In order to make the document self-contained, we start by reviewing basic compiler terminology.

\begin{Def}[Perfect and imperfect loop nests]
A loop nest is said to be \textit{perfect} when code appears only in the body of the innermost loop. Conversely, the presence of code between any pair of loops in a nest renders the loop nest \textit{imperfect}. 
\end{Def}

A straightforward property of perfect nests is that hoisting of invariant expressions from the innermost loop to the preheader (i.e., the block that precedes the entry point of the nest) is always safe. We will later make extensive use of this property.

% In the following, we should represent loops by their iteration variable.

\begin{Def}[Linear loop]
A loop $L$ defining the iteration space $I$ through the iteration variable $i$, or simply $L_i$, is \textit{linear} if all expressions appearing in the body of $L$ that use $i$ to access some memory locations are linear over $I$.
\end{Def}

In this work, we are particularly interested in the following class since it naturally arises from the math described in Section~\ref{sec:background}.

\begin{Def}[Perfect multilinear loop nest]
A \textit{perfect multilinear} loop nest of arity $n$ is a perfect nest composed of $n$ loops, in which all of the expressions appearing in the body of the innermost loop are linear in each loop $L_i$ separately.
\end{Def}

Note that nothing prevents a perfect multilinear loop nest from being enclosed by an outer, possibly imperfect, nest. Indeed, this loop structure will be at the centre of our study. We will later elaborate on this. First, we need to define optimality for perfect multilinear loop nests. To this purpose, it is convenient to introduce the notion of sharing.

\begin{Def}[Sharing]
A loop $L_i$ presents \textit{sharing} if it contains at least two expressions depending on $i$ that are symbolically identical. 
\end{Def}

At this point, we can present a simple yet fundamental result.

\begin{Prop}
\label{prop:share-removal}
Sharing in a perfect multilinear loop nest $LN = [L_{i_{0}}, L_{i_{1}}, ..., L_{i_{n-1}}]$ can always be eliminated.
\end{Prop}
\begin{proof}
The demonstration is by construction and exploits linearity. We want to transform $LN$ into $LN'$ such that there is no sharing is any $L_i \in LN'$. Starting from the innermost loop $L_{i_{n-1}}$, the expressions are ''flattened'' by expanding all products involving terms depending on $L{i_{n-1}}$. Being on the same level of the expression tree, such terms can then be factorized. Due to linearity, each factored product only has one term depending on $L_{i_{n-1}}$. The other terms, independent of $L_{i_{n-1}}$, are, by definition, loop-invariant, and as such can be hoisted at the level of $L_{i_{n-2}}$. This procedure can be applied recursively up to $L_{i_{0}}$: multilinearity allows factorization at each level; perfectness ensures hoisting is always safe.
\end{proof}

Based on this proposition, we define optimality as follows.

\begin{Def}[Optimality of a multilinear loop nest]
The synthesis of a multilinear loop nest is \textit{optimal} if the amount of operations performed in the innermost loop is minimum.
\end{Def}

In other words, this means there is no other nest synthesis able to decrease the number of operations in the innermost loop any further. Note that this definition of optimality does not take into account memory requirements. If the loop nest were memory-bound -- the ratio of operations to bytes transferred from memory to the CPU being too low -- then speaking of ''optimality'' would clearly make no sense. In the following, we assume to operate in a CPU-bound regime, in which the body of loop nests are characterized by arithmetic-intensive expressions. This suits the context of finite element integration.

A second result follows.
\begin{Prop}
\label{prop:optimal-mln}
An optimal synthesis for a perfect multilinear loop nest $LN$ can be found in ....
\end{Prop}
\begin{proof}
By construction. We start from the innermost loop $L_{i_{n-1}}$. Loop-dependent terms are logically grouped into $n$ disjoint sets $S_i$, each $S_i$ containing all terms depending on $L_i$. These sets are sorted in descending order based on their cardinality. By establishing a one-to-one mapping between set indices and loop indices, we produce a new loop permutation. The loop permutation is semantically correct because of perfectness. In this new order, loops are placed such that as going down the nest, $L_i$ is characterized by less unique terms than $L_{i-1}$. At this point, we can apply the sharing-removal procedure described in Proposition~\ref{prop:share-removal}. This renders $L_{i_{n-1}}$ no further improvable in the number of operations executed. In particular, the number of operations is equal to $\sharp S_{i_{n-1}} + (\sharp S_{i_{n-1}} - 1)$, in which the second term represents the cost of the summation.
\end{proof}

We now consider generic loop nests in which multilinearity applies to a subset of loops only. Consider the example in Figure~\ref{code:loopnest}. 

\begin{figure}[h]\begin{CenteredBox}
\lstinputlisting[basicstyle=\footnotesize\ttfamily]{listings/loopnest.code}
\end{CenteredBox}\caption{Loop nest example}\label{code:loopnest}\end{figure}

The imperfect nest $LN=[L_e, L_i, L_j, L_k]$ comprises a reduction loop $L_i$ and a perfect doubly nested loop $[L_j, L_k]$, which we assume to be multilinear. The right hand side of the statement computing the multidimensional array \texttt{A} is a generic expression \texttt{F} including standard arithmetic operations such as addition and multiplication. We observe that \texttt{F} might contain sub-expressions that depend on a subset of loops only and, as such, hoistable outside of $LN$ as long as data dependencies are preserved (note that $LN$ is imperfect). One could think of pre-evaluating the reduction, thus obtaining a decrease proportional to $M$ in the number of operations executed. However, finding or exposing reducible sub-expressions is, in general, challenging. Further, even though we assumed we could do so, the following issues should be addressed
\begin{itemize}
\item as opposed to what happens with hoisting in perfect multilinear loop nests, in this case the size of the temporaries would be proportional to the number of non-reduction loops crossed (in the example, \texttt{N$\cdot$O} for \texttt{ijk} sub-expressions, \texttt{L$\cdot$N$\cdot$O} for \texttt{eijk} ones). This might shift the loop nest from a CPU-bound to a memory-bound regime, which might be counter-productive for actual runtime;
\item the process by which multi-invariant sub-expressions are exposed could require transformations like expansion and factorization. Consequently, the save originating from the elimination of the reduction loop could be overwhelmed by a potentially larger increase in the number of operations in the body of $L_k$ (expansion could increase the number of operations, e.g. $A(B+C) = AB + AC$).
\end{itemize}

We then refine our definition of optimality for generic loop nests as follows

\begin{Def}[Optimality of a loop nest]
\label{def:optimality}
The synthesis of a loop nest is \textit{optimal} if, under a set of memory constraints \texttt{C}, the total amount of operations performed in all innermost loops is minimum.
\end{Def}

Clearly, this definition remains vague until instantiated in a certain model. Once the model is fixed, we can argue about the meaning of \texttt{C} and try to produce an algorithm to reach optimality. This is tackled in the next section.

\section{Synthesis of Optimal Loop Nests in Finite Element Integration}
\label{sec:optimal-syntesis}

In this section, we instantiate our definition of loop optimality in the domain of finite element integration. This requires elaborating on the junction between two different levels of abstraction: the math, in terms of the multilinear forms arising from the weak variational formulation of a problem, which we reviewed in Section~\ref{sec:background}; and the (partly multilinear) loop nests implementing such forms.

Our point of departure is the example loop nest in Figure~\ref{code:loopnest}. This loop nest is a simplified view of a typical bilinear form implementation. The $L_e$ loop represents iteration over the elements of a mesh; the $L_i$ loop derives from using numerical quadrature; the perfect loop nest $[L_j, L_k]$ implements the computation over the test and trial functions' degrees of freedom. We deliberately omitted useless portions of code to not hinder readability (e.g. matrix insertion) and to avoid tying our discussing to a specific form (e.g. \texttt{F} is unspecified).

Before expanding on optimality, we make a few observations about properties of our domain. 1) We know that $L >>> M, N, O$; that is, even with high order basis functions, the number of elements is typically order of magnitude larger than both the number of quadrature points and degrees of freedom. 2) The loop nest enclosed by the integration loop is always perfect and multilinear; this naturally descends from the translation of Equation~\ref{eq:quadrature} into a loop nest. 

% maybe say somewhere the order of the loop nest eijk is just the most natural --- the other present worse data locality for example
% change L into some other letters
% say somewhere that N and O are the number of dofs for test and trial functions

\subsection{Memory constraints}
\label{sec:mem-const}
The fact that the iteration space of $L_e$ is so larger than that of other loops suggests that we should be cautious when hoisting outside of $LN$. Imagine a time stepping loop $L_t$ wraps $LN$. Theoretically, one could think of identifying time-invariant sub-expressions that access both geometry and reference element terms, and hoist them out of $L_e$. Unless adopting complex engineering solutions (e.g. aggressive blocking), which are practically difficult to devise and maintain, this kind of code motion increases the working set by a factor $L$. It is our opinion, therefore, that the decrease in the number of operations would be completely overwhelmed by the bigger memory pressure.

A second, more general observation is that, for certain forms and discretizations, aggressive hoisting can make the working set exceed the size of "some level of local memory" (e.g. the last level of private cache on a conventional CPU, the shared memory on a GPU). We will provide precise details about this in the following sections. For the moment, and just as one of many possible examples, note that applying tensor contraction mode (see Section~\ref{sec:background}), which essentially means lifting code outside of $L_e$, requires a set of temporaries of size equal to $N\cdot O$; depending on the discretization employed it is not so uncommon to break the local memory threshold.  

Based upon these considerations, we impose two memory constraints
\begin{itemize}
\item The size of a temporary due to code motion cannot be bigger than the size of the multilinear loop nest iteration space ($N\cdot O$ for the bilinear form in the example). A corollary is that hoisting expressions involving geometry terms outside of $L_e$ is forbidden.
\item The total size of the hoisted temporaries cannot exceed a threshold \texttt{$T_H$}
\end{itemize}

\subsection{Minimizing the Operation Cost}
Definition~\ref{def:optimality} states that a necessary condition for a loop nest synthesis to be optimal is that the number of operations in all innermost loops is minimum. We now discuss how we can systematically achieve this.

Eliminating sharing from the multilinear loops does not suffice for nest optimality. In fact, we wonder whether the reduction imposed by $L_i$ could be pre-evaluated outside of $N$, as already suggested in~\ref{def:optimality}, thus potentially reducing the operation count.

To answer this question, we make use of a result -- the foundation of tensor contraction mode -- from~\citeN{Kirby:TC}. Essentially, multilinear forms can be seen as sums of monomials, each monomial being an integral over the equation domain of products (of derivatives) of functions from discrete spaces; such monomials can always be reduced to a product of two tensors (see Section~\ref{sec:background}). We interpret this result at the loop nest level: with the input as in Figure~\ref{code:loopnest}, we can always dissect \texttt{F} into distinct sub-expressions (the monomials). Each sub-expression is then factorized so as to split constant from $[L_i, L_j, L_k]$-dependent terms, the latter ones are hoisted outside of $LN$, and finally pre-evaluated. As part of this pre-evaluation, the reduction induced by $L_i$ vanishes. In the following, we simply refer to this special sort of code hoisting as "\textit{pre-evaluation}". 

The intuition of the main algorithm for optimal loop nest synthesis is shown in Figure~\ref{code:intuition}. 

% local optimization of monomials
% local sharing elimination
% global sharing elimination
% we exclude pathological cases in which different monomials lead to pre-evaluating the same tables 

\begin{figure}[h]
\begin{lstlisting}[basicstyle=\small\ttfamily, frame=single]
dissect the input expression into monomials
for each monomial M:
  C = estimate $\mbox{\texttt{ops}}$ after pre-evaluation
  NC = estimate $\mbox{\texttt{ops}}$ without pre-evaluation
  if C < NC and memory constraints satisfied:
    mark M as candidate for pre-evaluation
for each monomial M:
  if M does not share terms with M$'$, an unmarked monomial:
    extract M into a separate loop nest
    apply pre-evaluation to M
for each expression:
  remove sharing    
\end{lstlisting}
\caption{Intuition of the main algorithm}
\label{code:intuition}
\end{figure}

The point of departure consists of understanding the impact, as number of operations saved or introduced, of pre-evaluation. This is studied "locally"; that is, for each monomial, in isolation. If we estimate that, for a given monomial, pre-evaluation will decrease the operation count, then the corresponding sub-expression is extracted, a sequence of transformation steps -- involving expansion, factorization, code motion -- takes place (details in Section~\ref{sec:codegen}, and the evaluation eventually performed. The result is a set of $n$-dimensional tables (these can be seen as "slices" of the reference tensor at the math level), $n$ being the arity of the multilinear form. Identical tables are mapped to the same temporary. Eventually, sharing is removed from the resulting expressions by applying a procedure as described in Proposition~\ref{prop:optimal-mln}. 

The transformed loop nest is as in Figure~\ref{code:loopnest-opt}. 

\begin{figure}[h]\begin{CenteredBox}
\lstinputlisting[basicstyle=\footnotesize\ttfamily]{listings/loopnest_opt.code}
\end{CenteredBox}\caption{Optimized Loop nest example}\label{code:loopnest-opt}\end{figure}

Before elaborating on the profitability of pre-evaluation, we need to discuss under which conditions this approach, based on a "local analysis" of monomials, is optimal. 

\begin{Prop}
\label{prop:optimal-approach}
Consider an expression comprising a set of monomials $M$. Let $P$ be the set of pre-evaluated monomials, determined as described in Figure~\ref{code:intuition}, and let $Z$ be such that $Z = M \setminus P$. Assume that:
\begin{enumerate}
\item the cost function employed is optimal; that is, it predicts correctly the profitability of pre-evaluation
\item pre-evaluating distinct monomials does not produce identical tables
\item monomials in $P$ do not share terms
\end{enumerate} 
Then, the loop nest $LN$ is optimal under memory constraints $C$, once sharing is removed.
\end{Prop}
\begin{proof}
We first comment on the assumptions. 1) How to create an optimal cost function is discussed in Section~\ref{sec:op_count}. 2) A pathological case due to symmetries in basis functions, which in practice rarely happens. 3) This could occur in complex forms with several monomials; for simplicity, we ignore this situation (otherwise, a "global" analysis of the monomials would be required).

We distinguish two classes of loop nests rooted in $LN$: $[L_e, L_j, L_k]$, for the pre-evaluated monomials in $P$, and $[L_e, L_i, L_j, L_k]$, enclosing the remaining monomials in $Z$. Since they only differ for the presence of $L_i$, we relieve notation by omitting all shared loops when discussing operation counts. In particular, we use $I$ to refer to the iteration space size of $L_i$. The operation count of what we are proving to be the optimal $LN$ synthesis is, therefore, $LN_{ops} = LN_{ops_1} + LN_{ops_2} = \sum_{\alpha}^{\# P} p_{\alpha} + I \sum_{\beta}^{\# Z} z_{\beta}$, where $p_{\alpha}$ and $z_{\beta}$ represent the operation cost of monomials in $P$ and $Z$, respectively.

We start noting that, as explained in Section~\ref{sec:mem-const}, $C$ imposes constraints on hoisting. This narrows the proof to demonstrating the following: A) pre-evaluating any $Z_P: Z_P \subseteq Z$ would increase $LN_{ops}$; B) not pre-evaluating any $P_Z: P_Z \subseteq P$ would increase $LN_{ops}$.

A) We prove that $LN_{ops}' = LN_{ops_1}' + LN_{ops_2}' > LN_{ops}$. It is rather obvious that $LN_{ops_1}' \geq LN_{ops_1}$ (it is equal only if, trivially, $Z_P = \emptyset$). We note that if monomials in $Z_P$ share symbols with $\overline{Z} = Z \setminus Z_P$, then we have $LN_{ops_2}' = LN_{ops_2}$, so our statement is true. If, on the other hand, at least one monomial does not share any symbols, we obtain $LN_{ops_2}' < LN_{ops_2}$ or, equivalently, $LN_{ops_2}' = LN_{ops_2} - I \cdot\delta$. What we have to show now is that even by exposing more pre-evaluations, $LN_{ops_1}' \geq LN_{ops_1} +  I \cdot\delta$ holds. This is indeed the case since we rely on assumption 2, which ensures the uniqueness of the pre-evaluated tables (i.e., the absence of sharing) and, therefore, the optimality of $LN$. 

B) In absence of sharing, the statement is trivially true since we would have $LN_{ops_2}' > LN_{ops_2}$, with the cost function being optimal by assumption. Assumption 3 guarantees there can be no sharing within $P_Z$, which avoids subtle cases wherein pre-evaluation would result sub-optimal due to destroying sharing-removal opportunities. The last case we have to consider is when $p \in P_Z$ shares at least one term with $z \in Z$. This situation cannot actually occur by construction: all candidates for pre-evaluation sharing terms with monomials in $Z$ are "declassed" from $P$ to $Z$: the rationale is that we would have to pay anyway the presence of the shared terms in the innermost loop due to $z$, so aggregating $p$ in $Z$ does not increase the operation count. 
\end{proof}

\subsection{A-Priori Operation Counting}
\label{sec:op_count}
In this section we discuss the optimality of the cost function. 

%\item cost model for deciding what to do
%\item heuristic optimisation of constant and quadrature dependent expressions

% apply memory constraints: reduction to knapsack problem ... not implemented, we use greedy



\section{Code Generation}
\label{sec:codegen}

\begin{itemize}
\item Deficiencies of previous approaches
\item  Building block operations in coffee: 1) eliminating reuse, 2)  insights on the various tree algorithms employed
\end{itemize}

\subsection{Heuristic Optimization of Integration-dependent Expressions}
%insights that optimisation of loops outside of the multilinear nest can be tackled greedily and heuristically

\subsection{Low-level Optimization}
\label{sec:code-spec}
...

% MAYBE JUST A SUBSECTION, "INSIGHTS OF LOW-LEVEL OPTS ...."

\subsubsection{Avoiding Iteration over Zero-valued Blocks by Symbolic Execution}
\label{sec:zeros}
%TODO : say that the reasons to have zeros is to keep the design of the form compiler simple, which just "puts zeros" and "outline the assembly expression".

% say that we just don't kill al zero-columns and use indirections to not break vectorisation

% generic engine to remove zeros. Can be rows (RT elements), columns (derivatives, vector elements), blocks (vector elements, "TC mode"). Based on symbolic execution.
% rationale: contiguous in the innermost dimension are aggregated. contiguous in all other dimensions are not, unless complete overlap.
% tables shrunk and merged


\subsubsection{Padding and Data Alignment}
\label{sec:padding}
...

\subsubsection{Vector-promotion of Integration Quantities}
...

\section{Performance Evaluation}
\label{sec:perf-results}

\subsection{Experimental Setup}

Experiments were run on a single core of an Intel architecture, a Sandy Bridge I7-2600 CPU, running at 3.4GHz, 32KB L1 cache and 256KB L2 cache. The Intel \texttt{icc 14.2} compiler was used. The compilation flags used were \texttt{-O3, -xHost, -xAVX, -ip}.

We analyze the run-time performance of four fundamental real problems, which comprise the differential operators that are most common in finite element methods. In particular, our study includes problems based upon the Helmholtz and Poisson equations, as well as elasticity- and hyperelasticity-like forms. The Unified Form Language~\cite{UFL} specification for these forms, which is the domain specific language that both Firedrake and FEniCS use to express weak variational form, is available at~\cite{ufl-code}. 

We evaluate the \textit{speed ups} achieved by three sets of optimizations over the original code; that is, the code generated by the FEniCS Form Compiler when no optimizations are applied. In particular, we analyze the impact of the FEniCS Form Compiler's built-in optimizations (henceforth \texttt{ffc}), the impact of COFFEE's transformations as presented in~\cite{Luporini} (referred to as \texttt{fix}, in the following), and the effect of Expression Rewriting and Code Specialization as described in this work (henceforth \texttt{auto}, to denote the use of autotuning as described in Section~\ref{sec:autotune}). The \texttt{auto} values do not include the autotuner cost, which is commented aside in Section~\ref{sec:auto-analysis}. 

%This way, we also highlight the advances achieved over our previous work. 
%...TODO...: dire che i tempi sono la media di 3 runs

The values that we report include the cost of local assembly as well as the cost of matrix insertion. However, the unstructured mesh has been made small enough to fit the L3 cache, so as to minimize the ``noise'' due to any operations that are not part of the element matrix evaluation itself. However, it has been reiterated over and over (e.g.~\cite{quadrature1}) that as the complexity of a form increseas, the cost of local assembly becomes dominant. All codes were executed in the context of the Firedrake framework.

We vary several aspects of each form, which follows the approach and the notation of~\cite{quadrature1} and~\cite{Francis}
\begin{itemize}
\item The polynomial order of basis functions, $q \in \lbrace1, 2, 3, 4\rbrace$
\item The polynomial order of coefficient (or ``pre-multiplying'') functions, $p \in \lbrace1, 2, 3, 4\rbrace$
\item The number of coefficient functions $nf \in \lbrace0, 1, 2, 3\rbrace$
\end{itemize}
On the other hand, other aspects are fixed 
\begin{itemize}
\item The space of both basis and coefficient functions is Lagrange
\item The mesh is three-dimensional, made of tetrahedrons, for a total of 4374 cells
\end{itemize}

Figures~\ref{fig:helmholtz},~\ref{fig:elasticity},~\ref{fig:poisson}, and~\ref{fig:hyperelasticity}, which will be deeply commented in the next section, must be read as ``plots, or grids, of plots''. Each grid (figure) has two logical axes: $p$ varies along the horizontal axis, while $q$ varies along the vertical axis. The top-left plot in a grid shows speed ups for $[q=1, p=1]$; the plot on its right does the same for $[q=1, p=2]$, and so on. The diagonal of the grid shows plots for which basis and coefficient functions have same polynomial order, that is $q=p$. Therefore, a grid can be read in many different ways, which allows us to make structured considerations on the effect of the various optimizations. 

A plot reports speed-ups over non-optimized FEniCS-Form-Compiler-generated code. There are three groups of bars, each group referring to a particular version of the code (\texttt{ffc, fix, auto}). There are four bars per group: the leftmost bar corresponds to the case $nf = 0$, the one on its right to the case $nf = 1$, and so on. 


\subsection{Performance of Forms}
\label{sec:perf-results-forms}
%The four chosen forms allow us to perform an in-depth evaluation of different classes of optimizations for local assembly. We limit ourselves to analyzing the cost of computing element matrices, although all of the techniques presented in this paper are immediately extendible to the evaluation of local vectors. As anticipated, in the following we comment speed ups of \texttt{ffc}, \texttt{fix}, and \texttt{auto} over the non-optimized, FEniCS-Form-Compiler-generated code. 
%
%We first comment on results of general applicability. By looking at the various figures, we note there is a trend in COFFEE's optimizations to become more and more effective as $q$, $p$, and $nf$ increase. This is because most of the transformations applied aim at optimizing for arithmetic intensity and SIMD vectorization, which obviously have a strong impact when arrays and iteration spaces are large. The corner cases of this phenomenon are indeed $[q=1, p=1]$ and $[q=4, p=4]$. We also observe how \texttt{auto}, in almost all scenarios, outperforms all of the other variants. In particular, it is not a surprise that \texttt{auto} is faster than \texttt{fix}, since \texttt{fix} is one of the autotuner's tested variants, as explained in Section~\ref{sec:autotune}. This proves the quality of the work presented in this paper, which shows significant advances over~\cite{Luporini}. The reasons for which \texttt{auto} exceeds both original code and \texttt{ffc} are discussed for each specific problem next. Also, details on the ``optimal'' code variant determined by autotuning are given in Section~\ref{sec:auto-analysis}.

\paragraph{Helmholtz}
%\begin{figure}[t]  
%...
%%\includegraphics[scale=0.7]{perf-results/helmholtz}
%%\caption{Helmholtz results.}\label{fig:helmholtz}
%\end{figure}
%The results for the Helmholtz problem are provided in Figure~\ref{fig:helmholtz}. We observe that \texttt{ffc} slows the code down, especially for $q \geq 3$. This is a consequence of using indirection arrays in the generated code that, as explained in Section~\ref{sec:zeros}, prevent, among the other compiler optimizations, SIMD auto-vectorization. The \texttt{auto} version results in minimal performance improvements over \texttt{fix} when $nf=0$, unless $q=4$. This is due to the fact that if the loop over quadrature points is relatively small, then close-to-peak performance is obtainable through basic expression rewriting and code specialization; in this circumstance, generalized loop-invariant code motion and padding plus data alignment. The trend changes dramatically as $nf$ and $q$ increase: a more ample spectrum of transformations must be considered to find the optimal local assembly implementation. We will provide details about the selected transformations in the next section.

\paragraph{Elasticity}
%\begin{figure}[t]  
%...
%%\includegraphics[scale=0.7]{perf-results/elasticity}
%%\caption{Elasticity results.}\label{fig:elasticity}
%\end{figure}
%Figure~\ref{fig:elasticity} illustrates results for the Elasticity problem. This form uses a vector-valued space for the basis functions, so here transformations avoiding computation over zero-valued columns are of key importance. The \texttt{ffc} set of optimizations leads to notable improvements over the original code at $q=1$. The use of inderection arrays allows to phisically eliminate zero-valued columns at code generation time; as a consequence, different tabulated basis functions are merged into a single array. Therefore, despite the execution being purely scalar because of indirection arrays, the reduction in arithmetic intensity and register pressure imply improvement in performance. Nevertheless, \texttt{auto} remains in general the best choice, with gains over \texttt{ffc} that are wider as $p$ and $nf$ increase. 
%
%For $q \geq 2$, in \texttt{ffc} the lack of SIMD vectorization counterbalances the decrease in the number of floating point operations, leading to speed ups over the original code that only occasionally exceed 1$\times$. On the other hand, the successful application of the zero-avoidance optimization while preserving code specialization plays a key role for \texttt{auto}, resulting in much higher performance code especially at $q=2$ and $q=3$. 
%
%It is worth noting that speed ups of \texttt{auto} over \texttt{fix} decrease at $q=4$, particularly for low values of $p$. As we will discuss in Section~\ref{sec:auto-analysis}, this is because at $q=4$ the vector-register tiling transformation (in combination with loop unroll-and-jam) leads to the highest performance. In principle, vector-register tiling can be used in combination to the zero-avoidance technique; however, due to mere technical limitations, this is currently not supported in COFFEE. Once solved, we expect much higher speed ups in the $q=4$ regime as well.


\paragraph{Poisson}
%\begin{figure}[t]  
%...
%%\includegraphics[scale=0.7]{perf-results/poisson}
%%\caption{Poisson results.}\label{fig:poisson}
%\end{figure}
%
%In Figure~\ref{fig:poisson} we report speed ups of \texttt{ffc}, \texttt{fix}, and \texttt{auto} over the original code for the Poisson form. We note that, as a general trend, \texttt{ffc} exhibits drops in performance as $nf$ increases, notably when $nf=3$, for any values of $q$ and $p$. This is a consequence of the inherent complexity of the generated code. The way \texttt{ffc} performs loop-invariant code motion leads to the pre-computation of integration-dependent terms at the level of the integration loop, which are characterized by higher arithmetic intensity and redundant computation as $nf$ increases. Moreover, the absence of vectorization is another limiting factor.
%
%The \texttt{auto} variant generally shows the best performance. Significant improvements over \texttt{fix} are also achieved, notably as $q$, $p$ and $nf$ increase. As clarified in the next section, this is always due to a more aggressive expression rewriting in combination with the zero-avoidance technique.


\paragraph{Hyperelasticity}
%\begin{figure}[t]  
%...
%%\includegraphics[scale=0.7]{perf-results/hyperelasticity}
%%\caption{Hyperelasticity results.}\label{fig:hyperelasticity}
%\end{figure}
%
%Speed ups for the hyperelasticity form are shown in Figure~\ref{fig:hyperelasticity}. Experiments for $nf \geq 2$ could not be executed because of FEniCS-Form-Compiler's technical limitations. 
%
%For \texttt{auto}, massive speed ups for $q \geq 2$ are to be ascribed to aggressive and successful expression writing. Hyperelasticity problems are really compute-intensive, with thousands of operations being performed, so reductions in redundant and useless computation are crucial. Complex forms like hyperelasticity would benefit from further ``specialized'' optimizations: for example, it is a known technical limitation of COFFEE that, in some circumstances, less temporaries could (should) be generated and that hoisted code could (should) be suitably distributed over different loops to minimize register pressure (e.g. COFFEE could apply loop fission for obtaining significantly better register usage). We expect to obtain considerably faster code once such optimizations will be incorporated. 
%
%In the regime $q \geq 2$ and $nf=1$, peformance improvements are less pronounced moving from $p=1$ to $p=2$, although still significant; in particular, we notice a drop at $p=2$, followed by a raise up to $p=4$. It is worth observing that this effect is common to all sets of optimizations. The hypothesis is that this is due to the way coefficient functions are evaluated at quadrature points (identical in all configurations), which cannot be easily vectorized unless a change in storage layout and loops order is implemented in the code (abstract syntax tree) generator on top of COFFEE. 



\section{Conclusions}
\label{sec:conclusions}
...


% Bibliography
\bibliographystyle{ACM-Reference-Format-Journals}
\bibliography{biblio}


\medskip

\end{document}
